\documentclass[a4paper,10pt]{article}

\usepackage{Nikolai}
\usepackage[margin=0.75in]{geometry}

\title{Scientific Computing Project 4}
\author{Nikolai Plambech Nielsen\\LPK331}
\date{\today}

\begin{document}
	\maketitle
	\section*{Part 1}
	In the first part of the project we focus on simulating the spread of HIV in a population consisting of homosexual, bisexual and heterosexual males, along with heterosexual females.
	
	This results in a set of 4 coupled non-linear ordinary differential equations:
	\begin{align}
		\diff[d]{x_1}{t} &= a_1 x_1 (p_1-x_1) + a_2 x_2 ( p_1-x_1) - r_1 x_1 \\
		\diff[d]{x_2}{t} &= b_1 x_1 (p_2-x_2) + b_2 x_2 ( p_2-x_2) +  b_3 y(p_2-x_2) - r_2 x_2 \\
		\diff[d]{y}{t} &= c_1 x_2 (q-y) + a_2 z (q-y) - r_3 y \\
		\diff[d]{z}{t} &= d_1 y (r-z) + e_1 x_1 (r-z) - r_4 z	
	\end{align}
	where $ x_1, x_2, y, z $ are the infected populations of homosexual males, bisexual males, females and heterosexual males respectively. $ p_1, p_2, q, r $ is the total population (in same units as their respective variables). $ a_1, a_2, b_1,$ $ b_2, b_3, c_1, c_2, d_1 $ are infection rates between differing populations (due to sexual contact). $ e_1 $ is the infection rate for heterosexual males due to blood transfusion. $ r_1, r_2, r_3, r_4 $ are the death rates for given infected populations.
	
	To start with we set $ r_1 = r_2 = r_3 = r_4 = e_1 = 0$, ie, disregard death and blood transfusion. In this case we are dealing with coupled systems of logistic growth, where the total population is usually referred to the carrying capacity. In this case the equilibrium is when the infected population equals the total population (ie, when there are no more individuals to infect):
	\begin{equation}
		x_1 = p_1, \quad x_2=p_2, \quad y=q, \quad z=r.
	\end{equation}
	
	Adding the blood transfusion term (ie $ e_1 > 0 $) does not change the nature of the systems of equations - it will just increase the rate of infection for $ z $. The maximum increase in infection rate possible is just $ e_1p_1 $ - when $ x_1 $ is at a maximum.
	
	For the whole assignment we take $ a_1=10, a_2=b_1=5 $ and all other infection rates to be unity. That leaves $ e_1 $ as a free parameter, along with $ r_1, r_2, r_3, r_4 $, the death rates.
	
	We use the initial conditions of $ x_1 = 0.01 $ and $ x_2=y=z =0 $.

	We will simulate the system using both simple forward Euler integration and the 4th order Runge-Kutta method.
	
	\subsection*{No death or blood transfusion}
	Running the simulation with no blood transfusion or death we assume that the simulation reaches an equilibrium at the total populations. In figure \ref{fig:no_death} the simulation is run with both the methods, using $ \Delta t = 5\D 10^{-4} $ and running for 350 iterations.
	\begin{figure}[H]
		\centering
		\includegraphics[width=0.7\linewidth]{no_death_or_transfusion.pdf}
		\caption{Simulation using both forward Euler and RK4. No death terms or transfusion present.}
		\label{fig:no_death}
	\end{figure}
	Lastly we also check the convergence upon the expected values:
	\begin{table}[H]
		\centering
		\begin{tabular}{c|c|c|c|c}
			Method & $ x_1 $ & $ x_2 $ & $ y $ & $ z $ \\
			\hline
			Euler & 4.98488725 & 4.99962777 & 99.94163087 & 99.9188132  \\
			RK4 & 4.98554621 & 4.99963704 & 99.94619188 & 99.92539926 \\
			\hline \hline
			Expected & 5 & 5 & 100 & 100
		\end{tabular}
		\caption{Final values for all variables.}
		\label{tab:final_vals1}
	\end{table}
	And sure enough, they are all close to the expected values, with the result for RK4 being closer than that for Euler (due to the smaller error rate for RK4).	
	
	\subsection*{With blood transfusions}
	Next we add the blood transfusion term to the last equation. We plot the result for 101 values of $ e_1 $ between 0 and 1 (0, 0.01, 0.02, etc), and with all other parameters being equal to those before, using RK4. We also add contours to the plot at the values 0, 5, ..., 100. To verify that the transfusion term does not affect the asymptotic behaviour we also plot the last value of $ z $ for each value of $ e_1 $. The result is plotted in figure \ref{fig:transfusions}. 
	\begin{figure}[H]
		\centering
		\includegraphics[width=0.7\linewidth]{transfusions.pdf}
		\caption{Left: $ z $ as a function of time and blood transfusion rate. Contours are added to the plot at values 0, 5, ..., 100. Right: the last value of $ z $ as a function of blood transfusion rate.}
		\label{fig:transfusions}
	\end{figure}
	The final value of $ z $ is slightly different for different blood transfusion rates, though all close to the expected value of 100. The discrepancy stems from the fact that the blood transfusion increases the derivative of $ z $ slightly, so the convergence happens faster for higher values of $ e_1 $. This is also evident in the contours on the left plot.
	
	\subsection*{With deaths}
	Lastly we also add deaths to the system by adding an exponential decay to each equation.
	
	\newpage
	\section*{Part 2}
	For the second part we solve the reaction-diffusion equations:
	\begin{align}
		\diff{p}{t} &= D_p \nabla^2 p + p^2q+C-(K+1)p \\
		\diff{q}{t} &= D_q \nabla^2 q - p^2q+Kp
	\end{align}
	Where we diffusion in the first term, reaction in the second, and then some feeding and killing terms. We solve the equations on a square: $(x,y) \in \{0\leq x \leq 40, 0\leq y\leq 40\} $, with no-flux boundary conditions:
	\begin{equation}\label{key}
		\diff{p}{x}\bigg|_{y=0} = \diff{p}{x}\bigg|_{y=40} = \diff{p}{y}\bigg|_{x=0} = \diff{p}{y}\bigg|_{x=40} = 0
	\end{equation}
	and likewise for $ q $. For the initial condition we set
	\begin{align}\label{key}
		p(x,y,0) &= \begin{cases}
		C+0.1 & 10 \leq x \leq 30\  \wedge \ 10 \leq y \leq 30 \\
		0 & \text{otherwise}
		\end{cases} \\
		q(x,y,0) &= \begin{cases}
		K/C+0.2 & 10 \leq x \leq 30\  \wedge \ 10 \leq y \leq 30 \\
		0 & \text{otherwise}
		\end{cases}
	\end{align}
	So there is a constant concentration of $ p $ and $ q $ in a centre square, and no concentration otherwise.
	
	The goal is to simulate the system until $ t=2000 $ and show the results, for different values of $ K $: $ K \in [7,8,9,10,11,12] $, with $ D_p = 1 $, $ D_q=8 $ and $ C=4.5 $
	
	\subsection*{Implementation}
	We use a forward difference in time and a centred difference in space to discretize the system. For $ p $ we get
	\begin{align*}
		\frac{p(x, y, t+\Delta t) - p(x,y,t)}{\Delta t} &= D_p \frac{1}{h^2}\Big(p(x+h, y, t) + p(x-h, y, t) + p(x, y+h, t) + p(x, y-h, t) - 4p(x, y, t)\Big) \\
		&+ p(x,y,t)^2q(x,y,t) + C - (K+1) p(x,y,t)
	\end{align*}
	and for $ q $
	\begin{align*}
		\frac{q(x, y, t+\Delta t) - q(x,y,t)}{\Delta t} &= D_q \frac{1}{h^2}\Big(q(x+h, y, t) +q(x-h, y, t) + q(x, y+h, t) + q(x, y-h, t) - 4q(x, y, t)\Big) \\
		&- p(x,y,t)^2q(x,y,t) +K p(x,y,t)
	\end{align*}
	where $ h $ is the spatial grid spacing and $ \Delta t $ is the temporal grid spacing. The updating formulas are trivial from this point:
	\begin{align}
	p(x,y,t+\Delta t) &=p(x,y,t) + \Delta t \bigg[ D_p \frac{1}{h^2}\Big(p(x+h, y, t) + p(x-h, y, t) + p(x, y+h, t) + p(x, y-h, t) - 4p(x, y, t)\Big) \nonumber \\
	&+ p(x,y,t)^2q(x,y,t) + C - (K+1) p(x,y,t)\bigg] \label{eq:p_update}\\
	q(x,y,t+\Delta t) &=q(x,y,t) + \Delta t \bigg[ D_q \frac{1}{h^2}\Big(q(x+h, y, t) +q(x-h, y, t) + q(x, y+h, t) + q(x, y-h, t) - 4q(x, y, t)\Big) \nonumber\\
	&- p(x,y,t)^2q(x,y,t) +K p(x,y,t) \bigg] \label{eq:q_update}
	\end{align}
	To handle the boundary conditions, we pad the computational domain with a set of ghost nodes, so the values of $ p $ and $ q $ are stored for $ -h \leq x,y \leq 40+h $. The domain nodes use the regular updating formula, and then the ghost nodes use a separate updating scheme. This is based on a forward difference approximation on the boundary nodes:
	\begin{equation}
		\diff{p}{x}\bigg|_{x=0} \approx \frac{p(h, y, t) - p(0, y, t)}{h} 
	\end{equation}
	giving us the updating formula for the ghost nodes on this boundary:
	\begin{equation}\label{eq:ghosts}
		p(-h,y,t) = p(0,y,t)
	\end{equation}
	This all means that to update the left edge, we set the ghost nodes to be equal to the first column of inner nodes. We set the ghosts on the right edge equal to the last column of the inner nodes, and so on for the other edges, and also for $ q $. The algorithm is thus:
	\begin{enumerate}
		\item Initialize system (observe initial condition, etc.)
		\item Update domain nodes using updating equations \ref{eq:p_update} and \ref{eq:q_update}
		\item Update ghost nodes to observe boundary conditions, using equation \ref{eq:ghosts} or similar
		\item Repeat steps 2 and 3 until the time $ T=2000 $ is reached.
	\end{enumerate}
	Lastly we need to specify $ h $ and $ \Delta t $. I set the number of points to be $ N_x=121 $, which gives $ h = L/(N_x-1) = (40-0)/(121-1) =  1/3 $. For $ \Delta t $ I use a modification of the bound for the diffusion equation ($ \Delta t \leq h^2/2c $). I let $ c = \max (D_p, D_q, C, K) $ (which is essentially just $ \max(D_q, K) $) and choose
	\begin{equation}\label{key}
		\Delta t = \frac{h^2}{4c}
	\end{equation}
	This might be a conservative bound, but using Numpy operations, the compute time is a couple of minutes, which is acceptable to me.
	
\end{document}