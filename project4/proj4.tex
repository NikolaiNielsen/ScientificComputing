\documentclass[a4paper,10pt]{article}

\usepackage{Nikolai}
\usepackage[margin=0.75in]{geometry}

\title{Scientific Computing Project 4}
\author{Nikolai Plambech Nielsen\\LPK331}
\date{\today}

\begin{document}
	\maketitle
	\section*{Part 1}
	In the first part of the project we focus on simulating the spread of HIV in a population consisting of homosexual, bisexual and heterosexual males, along with heterosexual females.
	
	This results in a set of 4 coupled non-linear ordinary differential equations:
	\begin{align}
		\diff[d]{x_1}{t} &= a_1 x_1 (p_1-x_1) + a_2 x_2 ( p_1-x_1) - r_1 x_1 \\
		\diff[d]{x_2}{t} &= b_1 x_1 (p_2-x_2) + b_2 x_2 ( p_2-x_2) +  b_3 y(p_2-x_2) - r_2 x_2 \\
		\diff[d]{y}{t} &= c_1 x_2 (q-y) + a_2 z (q-y) - r_3 y \\
		\diff[d]{z}{t} &= d_1 y (r-z) + e_1 x_1 (r-z) - r_4 z	
	\end{align}
	where $ x_1, x_2, y, z $ are the infected populations of homosexual males, bisexual males, females and heterosexual males respectively. $ p_1, p_2, q, r $ is the total population (in same units as their respective variables). $ a_1, a_2, b_1,$ $ b_2, b_3, c_1, c_2, d_1 $ are infection rates between differing populations (due to sexual contact). $ e_1 $ is the infection rate for heterosexual males due to blood transfusion. $ r_1, r_2, r_3, r_4 $ are the death rates for given infected populations.
	
	To start with we set $ r_1 = r_2 = r_3 = r_4 = e_1 = 0$, ie, disregard death and blood transfusion. In this case we are dealing with coupled systems of logistic growth, where the total population is usually referred to the carrying capacity. In this case the equilibrium is when the infected population equals the total population (ie, when there are no more individuals to infect):
	\begin{equation}
		x_1 = p_1, \quad x_2=p_2, \quad y=q, \quad z=r.
	\end{equation}
	
	Adding the blood transfusion term (ie $ e_1 > 0 $) does not change the nature of the systems of equations - it will just increase the rate of infection for $ z $. The maximum increase in infection rate possible is just $ e_1p_1 $ - when $ x_1 $ is at a maximum.
	
	For the whole assignment we take $ a_1=10, a_2=b_1=5 $ and all other infection rates to be unity. That leaves $ e_1 $ as a free parameter, along with $ r_1, r_2, r_3, r_4 $, the death rates.
	
	We use the initial conditions of $ x_1 = 0.01 $ and $ x_2=y=z =0 $.

	We will simulate the system using both simple forward Euler integration and the 4th order Runge-Kutta method.
	
	\subsection*{No death or blood transfusion}
	Running the simulation with no blood transfusion or death we assume that the simulation reaches an equilibrium at the total populations. In figure \ref{fig:no_death} the simulation is run with both the methods, using $ \Delta t = 5\D 10^{-4} $ and running for 350 iterations.
	\begin{figure}[H]
		\centering
		\includegraphics[width=0.7\linewidth]{no_death_or_transfusion.pdf}
		\caption{Simulation using both forward Euler and RK4. No death terms or transfusion present.}
		\label{fig:no_death}
	\end{figure}
	Lastly we also check the convergence upon the expected values:
	\begin{table}[H]
		\centering
		\begin{tabular}{c|c|c|c|c}
			Method & $ x_1 $ & $ x_2 $ & $ y $ & $ z $ \\
			\hline
			Euler & 4.98488725 & 4.99962777 & 99.94163087 & 99.9188132  \\
			RK4 & 4.98554621 & 4.99963704 & 99.94619188 & 99.92539926 \\
			\hline \hline
			Expected & 5 & 5 & 100 & 100
		\end{tabular}
		\caption{Final values for all variables.}
		\label{tab:final_vals1}
	\end{table}
	And sure enough, they are all close to the expected values, with the result for RK4 being closer than that for Euler (due to the smaller error rate for RK4).	
	
	\subsection*{With blood transfusions}
	Next we add the blood transfusion term to the last equation. We plot the result for 101 values of $ e_1 $ between 0 and 1 (0, 0.01, 0.02, etc), and with all other parameters being equal to those before, using RK4. We also add contours to the plot at the values 0, 5, ..., 100. To verify that the transfusion term does not affect the asymptotic behaviour we also plot the last value of $ z $ for each value of $ e_1 $. The result is plotted in figure \ref{fig:transfusions}. 
	\begin{figure}[H]
		\centering
		\includegraphics[width=0.7\linewidth]{transfusions.pdf}
		\caption{Left: $ z $ as a function of time and blood transfusion rate. Contours are added to the plot at values 0, 5, ..., 100. Right: the last value of $ z $ as a function of blood transfusion rate.}
		\label{fig:transfusions}
	\end{figure}
	The final value of $ z $ is slightly different for different blood transfusion rates, though all close to the expected value of 100. The discrepancy stems from the fact that the blood transfusion increases the derivative of $ z $ slightly, so the convergence happens faster for higher values of $ e_1 $. This is also evident in the contours on the left plot.
	
	\subsection*{With deaths}
	Lastly we also add deaths to the system by adding an exponential decay to each equation. This gives us an extra 4 parameters to play with, along with the transfusion rate.
	
	The effect of the death parameters is to add a competing decay to the growth, causing a new equilibrium to appear. If we vary only a single of these parameters at a time we can easily find an analytical expression for the last parameters equilibrium value. Varying $ r_1 $ whilst keeping the others at 0 gives the equilibria:
	\begin{equation}
		x_1^* = \frac{-\beta - \sqrt{\beta^2 - 4\alpha \gamma}}{2 \alpha}, \quad \alpha=-a_1, \beta = a_1p_1-a_2p_2-r_1, \gamma =a_2p_2p_1
	\end{equation}
	with $ x_2^* = p_2 $, $ y^* = q $, $ z^* = r $. Similarly varying $ r_2 $ gives (with $ x_1^* = p_1 $, $ y^* = q $, $ z^* = r $)
	\begin{equation}
		x_2^* = \frac{-\beta - \sqrt{\beta^2 - 4\alpha \gamma}}{2 \alpha}, \quad \alpha=-b_2, \beta = b_2p_2-b_1p_1-b_3q-r_2, \gamma =b_1p_1p_2 + b_3qp_2
	\end{equation}
	for $ r_3 $ we get
	\begin{equation}
		y^* = \frac{c_1p_2q+c_2rq}{c_1p_2 + c_2r+r_3}
	\end{equation}
	and lastly for $ r_4 $ we get
	\begin{equation}
		z^* = \frac{d_1 q r + e_1p_1r}{d_1q+e_1p_1+r_4}
	\end{equation}
	We run each simulation for 151 different values of $ r_i $ in the intervals
	\begin{equation}
		r_1 \in [0,3], \quad r_2 \in [0, 20], \quad r_3 \in [0, 60] \quad r_4 \in [0, 60]
	\end{equation}
	and plot the analytical result along with the value at the last time step (500 time steps, with $ \Delta t = 5\D10^{-4} $ using RK4). We also plot the relative difference between the analytical and simulated result. This is all seen in figure \ref{fig:deaths1}. Sure enough, the analytical and simulated results differ only slightly, with a relative difference on the order of magnitude of $ 10^{-5} $ in any of the simulations.
	
	\begin{figure}[H]
		\centering
		\includegraphics[width=0.7\linewidth]{Deaths.pdf}
		\caption{Equilibria values for the populations, as a function of their death rates. In each experiment we only vary one parameter. All other death rates are kept at 0. On the right the relative difference between the analytical and simulated result is plotted.}
		\label{fig:deaths1}
	\end{figure}

	Lastly, we examine the effect of varying all four death parameters at once. We generate a set of 4 random death parameters (in the same interval as before) 20 times, and run the simulation for each of these sets of parameters.
	
	To estimate the equilibrium positions we also employ a multivariate Newtons method for solving nonlinear equations, with a tolerance of RMS between iterates of $ 10^{-6} $. The results are tabulated below in table \ref{tab:deaths}.
	
	\begin{table}[H]
		\centering
		\begin{tabular}{c|c|c|c|c}
			Parameter & $ x_1 $ & $ x_2 $ & $ y $ & $ z $ \\
			\hline
			\hline
			Death Rate & 0.229 & 1.944 & 6.433 & 29.497  \\
			Analytical & 4.985 & 4.922 & 92.623 & 75.846  \\
			Simulated & 4.985 & 4.922 & 92.622 & 75.845  \\
			Relative error & $ -1.50\D10^{-5} $ & $ -6.06\D10^{-7} $ & $ -1.29\D10^{-5} $ & $ -9.68\D10^{-6} $ \\
			\hline
			\hline
			Death Rate & 1.395 & 3.031 & 50.520 & 14.019  \\
			Analytical & 4.907 & 4.841 & 63.178 & 81.840  \\
			Simulated & 4.906 & 4.841 & 63.167 & 81.819  \\
			Relative error & $-2.69\D10^{-5}$ & $-1.79\D10^{-5}$ & $-1.75\D10^{-4}$ & $-2.547\D10^{-4}$  \\
			\hline
			\hline
			Death Rate & 2.654 & 4.428 & 0.532 & 8.029  \\
			Analytical & 4.823 & 4.833 & 99.456 & 92.530  \\
			Simulated & 4.823 & 4.833 & 99.456 & 92.530  \\
			Relative error & $-1.80\D10^{-5}$ & $-3.38\D10^{-7}$ & $-8.38\D10^{-7}$ & $-7.17\D10^{-7}$  \\
			\hline
			\hline
			Death Rate & 0.070 & 17.688 & 51.153 & 31.053  \\
			Analytical & 4.995 & 4.152 & 57.450 & 64.913  \\
			Simulated & 4.995 & 4.151 & 57.404 & 64.852  \\
			Relative error & $-4.82\D10^{-5}$ & $-2.36\D10^{-4}$ & $-7.92\D10^{-4}$ & $-9.38\D10^{-4}$
		\end{tabular}
	\caption{4 different simulations, with random death rates. Showing the analytical and simulated equilibrium, along with the relative error.}
	\label{tab:deaths}
	\end{table}
	
	
	\newpage
	\section*{Part 2}
	For the second part we solve the reaction-diffusion equations:
	\begin{align}
		\diff{p}{t} &= D_p \nabla^2 p + p^2q+C-(K+1)p \\
		\diff{q}{t} &= D_q \nabla^2 q - p^2q+Kp
	\end{align}
	Where we diffusion in the first term, reaction in the second, and then some feeding and killing terms. We solve the equations on a square: $(x,y) \in \{0\leq x \leq 40, 0\leq y\leq 40\} $, with no-flux boundary conditions:
	\begin{equation}
		\diff{p}{x}\bigg|_{y=0} = \diff{p}{x}\bigg|_{y=40} = \diff{p}{y}\bigg|_{x=0} = \diff{p}{y}\bigg|_{x=40} = 0
	\end{equation}
	and likewise for $ q $. For the initial condition we set
	\begin{align}
		p(x,y,0) &= \begin{cases}
		C+0.1 & 10 \leq x \leq 30\  \wedge \ 10 \leq y \leq 30 \\
		0 & \text{otherwise}
		\end{cases} \\
		q(x,y,0) &= \begin{cases}
		K/C+0.2 & 10 \leq x \leq 30\  \wedge \ 10 \leq y \leq 30 \\
		0 & \text{otherwise}
		\end{cases}
	\end{align}
	So there is a constant concentration of $ p $ and $ q $ in a centre square, and no concentration otherwise.
	
	The goal is to simulate the system until $ t=2000 $ and show the results, for different values of $ K $: $ K \in [7,8,9,10,11,12] $, with $ D_p = 1 $, $ D_q=8 $ and $ C=4.5 $
	
	\subsection*{Implementation}
	We use a forward difference in time and a centred difference in space to discretize the system. For $ p $ we get
	\begin{align*}
		\frac{p(x, y, t+\Delta t) - p(x,y,t)}{\Delta t} &= D_p \frac{1}{h^2}\Big(p(x+h, y, t) + p(x-h, y, t) + p(x, y+h, t) + p(x, y-h, t) - 4p(x, y, t)\Big) \\
		&+ p(x,y,t)^2q(x,y,t) + C - (K+1) p(x,y,t)
	\end{align*}
	and for $ q $
	\begin{align*}
		\frac{q(x, y, t+\Delta t) - q(x,y,t)}{\Delta t} &= D_q \frac{1}{h^2}\Big(q(x+h, y, t) +q(x-h, y, t) + q(x, y+h, t) + q(x, y-h, t) - 4q(x, y, t)\Big) \\
		&- p(x,y,t)^2q(x,y,t) +K p(x,y,t)
	\end{align*}
	where $ h $ is the spatial grid spacing and $ \Delta t $ is the temporal grid spacing. The updating formulas are trivial from this point:
	\begin{align}
	p(x,y,t+\Delta t) &=p(x,y,t) + \Delta t \bigg[ D_p \frac{1}{h^2}\Big(p(x+h, y, t) + p(x-h, y, t) + p(x, y+h, t) + p(x, y-h, t) - 4p(x, y, t)\Big) \nonumber \\
	&+ p(x,y,t)^2q(x,y,t) + C - (K+1) p(x,y,t)\bigg] \label{eq:p_update}\\
	q(x,y,t+\Delta t) &=q(x,y,t) + \Delta t \bigg[ D_q \frac{1}{h^2}\Big(q(x+h, y, t) +q(x-h, y, t) + q(x, y+h, t) + q(x, y-h, t) - 4q(x, y, t)\Big) \nonumber\\
	&- p(x,y,t)^2q(x,y,t) +K p(x,y,t) \bigg] \label{eq:q_update}
	\end{align}
	To handle the boundary conditions, we pad the computational domain with a set of ghost nodes, so the values of $ p $ and $ q $ are stored for $ -h \leq x,y \leq 40+h $. The domain nodes use the regular updating formula, and then the ghost nodes use a separate updating scheme. This is based on a forward difference approximation on the boundary nodes:
	\begin{equation}
		\diff{p}{x}\bigg|_{x=0} \approx \frac{p(h, y, t) - p(0, y, t)}{h} 
	\end{equation}
	giving us the updating formula for the ghost nodes on this boundary:
	\begin{equation}\label{eq:ghosts}
		p(-h,y,t) = p(0,y,t)
	\end{equation}
	This all means that to update the left edge, we set the ghost nodes to be equal to the first column of inner nodes. We set the ghosts on the right edge equal to the last column of the inner nodes, and so on for the other edges, and also for $ q $. The algorithm is thus:
	\begin{enumerate}
		\item Initialize system (observe initial condition, etc.)
		\item Update domain nodes using updating equations \ref{eq:p_update} and \ref{eq:q_update}
		\item Update ghost nodes to observe boundary conditions, using equation \ref{eq:ghosts} or similar
		\item Repeat steps 2 and 3 until the time $ T=2000 $ is reached.
	\end{enumerate}
	Lastly we need to specify $ h $ and $ \Delta t $. I set the number of points to be $ N_x=201 $, which gives $ h = L/(N_x-1) = (40-0)/(201-1) =  1/5 $. For $ \Delta t $ I use a modification of the bound for the diffusion equation ($ \Delta t \leq h^2/2c $). I let $ c = \max (D_p, D_q, C, K) $ (which is essentially just $ \max(D_q, K) $) and choose
	\begin{equation}
		\Delta t = \frac{h^2}{4.8c}
	\end{equation}
	Each simulation takes around an hour to complete with this fine resolution grid, especially due to the constraint on $ \Delta t $, yielding between 1.9 and 2.8 million iterations.
	
	Though for a lower resolution of $ N_x = 41 $ with $ h = 1 $ the simulation takes a couple of minutes, which seems acceptable.
	
	The results are shown in figures \ref{fig:RD1} and \ref{fig:RD2}, for each of the different values of $ K $:
	\begin{figure}[H]
		\centering
		\includegraphics[width=0.7\linewidth]{RD_K7.pdf}
		\includegraphics[width=0.7\linewidth]{RD_K8.pdf}
		\includegraphics[width=0.7\linewidth]{RD_K9.pdf}
		\caption{Forward Euler integration for the Reaction Diffusion system. With parameters $ D_p=1, D_q = 8, C=4.5 $ and $ K \in \{7,8,9\} $}
		\label{fig:RD1}
	\end{figure}

	\begin{figure}[H]
		\centering
		\includegraphics[width=0.7\linewidth]{RD_K10.pdf}
		\includegraphics[width=0.7\linewidth]{RD_K11.pdf}
		\includegraphics[width=0.7\linewidth]{RD_K12.pdf}
		\caption{Forward Euler integration for the Reaction Diffusion system. With parameters $ D_p=1, D_q = 8, C=4.5 $ and $ K \in \{10,11,12\} $}
		\label{fig:RD2}
	\end{figure}
	\subsection*{Crank-Nicholson}
	Lastly I tried getting the Crank-Nicholson method to work. This method combines the forward and backwards Euler method, sort of averaging the two. If $ \dot{y}(x,t) = f(y(x, t), x, t) $ is the equation to solve, then the Crank-Nicholson method aims to solve the system
	\begin{equation}
		\frac{y(x, t+\Delta t) - y(x, t)}{\Delta t}  = \frac{1}{2} \bb{f(y(x, t), x, t) + f(y(x, t+\Delta t), x, t)}
	\end{equation}
	where the first term corresponds to a regular forwards Euler integration, and the second corresponds to a backwards Euler. This entails solving a system of algebraic equations per time step (due to the finite difference approximation).

	In our case, we let $ p_{i,j}^n $ denote $ p(x_i, y_j, t_n) $ (and similarly with $ q $). Then the differential equations become
	\begin{align*}
		\frac{p_{i,j}^{n+1} - p_{i,j}^{n+1}}{\Delta t} &= \frac{1}{2} \big[D_p \grad^2 p_{i,j}^{n} + (p_{i,j}^{n})^2 q_{i,j}^{n} + c - (K-1) p_{i,j}^{n}\big] + \frac{1}{2} \big[D_p \grad^2 p_{i,j}^{n+1} + (p_{i,j}^{n+1})^2 q_{i,j}^{n+1} + c - (K-1) p_{i,j}^{n+1}\big] \\
		\frac{q_{i,j}^{n+1} - q_{i,j}^{n+1}}{\Delta t} &= \frac{1}{2} \big[D_q \grad^2 q_{i,j}^{n} - (p_{i,j}^{n})^2 q_{i,j}^{n} + K p_{i,j}^{n}\big] + \frac{1}{2} \big[D_q \grad^2 q_{i,j}^{n+1} - (p_{i,j}^{n+1})^2 q_{i,j}^{n+1} + K p_{i,j}^{n+1}\big]
	\end{align*}
	Where the Laplacians are given as in the updating formula for the forward Euler method. These are the equations we need to solve for each time step. We rearrange these and employ Newtons method for multivariate expressions to solve them:
	\begin{align*}
		f &= p_{i,j}^{n+1} -  \frac{1}{2} \big[D_p \grad^2 p_{i,j}^{n+1} + (p_{i,j}^{n+1})^2 q_{i,j}^{n+1} + c - (K-1) p_{i,j}^{n+1}\big] - p_{i,j}^{n} - \frac{1}{2} \big[D_p \grad^2 p_{i,j}^{n} + (p_{i,j}^{n})^2 q_{i,j}^{n} + c - (K-1) p_{i,j}^{n}\big] \\
		&= p_{i,j}^{n+1} -  \frac{1}{2} \big[D_p \grad^2 p_{i,j}^{n+1} + (p_{i,j}^{n+1})^2 q_{i,j}^{n+1} + c - (K-1) p_{i,j}^{n+1}\big] + \alpha(p_{i,j}^n, q_{i,j}^n) = 0 \\
		g &= q_{i,j}^{n+1} -  \frac{1}{2} \big[D_q \grad^2 q_{i,j}^{n+1} - (p_{i,j}^{n+1})^2 q_{i,j}^{n+1} + K p_{i,j}^{n+1}\big] - q_{i,j}^{n} - \frac{1}{2} \big[D_q \grad^2 q_{i,j}^{n} - (p_{i,j}^{n})^2 q_{i,j}^{n} + K p_{i,j}^{n}\big] \\
		&= q_{i,j}^{n+1} -  \frac{1}{2} \big[D_q \grad^2 q_{i,j}^{n+1} - (p_{i,j}^{n+1})^2 q_{i,j}^{n+1} + K p_{i,j}^{n+1}\big] + \beta(p_{i,j}^n, q_{i,j}^n) = 0
	\end{align*}
	where the constants $ \alpha $ and $ \beta $ only depend on the values of $ p $ and $ q $ for the previous time steps. These can then be precomputed for each time step. 
	
	This gives us the objective function for all domain nodes. For the ghost nodes we just let $ f=p_{i,j}^{n+1} - p_{i,j}^{n}$, and likewise for $ g $ - ie we do not update them. The actual handling of boundary conditions is done as in the forwards Euler case. (in practice, we update these for each iteration of Newtons method, to make sure that the solution is correct).
	
	Lastly we just need the Jacobian for Newtons method. For each equation this matrix has at most 6 non-zero entries:
	\begin{align*}
		\diff{f}{p_{i,j}^{n+1}} &= 1+\Big(K+1+\frac{4D_p}{h^2}\Big) \frac{\Delta t}{2} - p_{i,j}^{n+1} q_{i,j}^{n+1}\Delta t, \quad \diff{f}{q_{i,j}^{n+1}} = \frac{\Delta t}{2} (p_{i,j}^{n+1})^2 \\
		\diff{f}{p_{i+1,j}^{n+1}} &= \diff{f}{p_{i-1,j}^{n+1}} = \diff{f}{p_{i,j+1}^{n+1}} = \diff{f}{p_{i,j-1}^{n+1}} = -\frac{D_p \Delta t}{2}
	\end{align*}
	and
	\begin{align*}
	\diff{g}{q_{i,j}^{n+1}} &= 1+\frac{2D_p \Delta t}{h^2} + \frac{\Delta t}{2} (p_{i,j}^{n+1})^2, \quad \diff{g}{q_{i,j}^{n+1}} = p_{i,j}^{n+1} q_{i,j}^{n+1} \Delta t \\
	\diff{g}{q_{i+1,j}^{n+1}} &= \diff{g}{q_{i-1,j}^{n+1}} = \diff{g}{q_{i,j+1}^{n+1}} = \diff{g}{q_{i,j-1}^{n+1}} = -\frac{D_q \Delta t}{2}
	\end{align*}
	And of course for the ghost nodes we just have the diagonal entries equal to 1.
	
	We see that the Jacobian can be split into a constant part and a part dependent on the current guess for $ p $ and $ q $. The constant part has a slightly more complicated form than the non-constant, and this matrix is populated in a for-loop. The second part only has non-zero values along 3 diagonals: The main diagonal, and ones offset by $ \pm (N_x+2)^2$ (those coupling $ p_{i,j}^{n+1} $ and $ q_{i,j}^{n+1} $). These can easily be created on the fly.
	
	Because there are $ 2 (N_x+2)^2$ unknowns to solve for, we use sparse matrices (to store just the floating point numbers with no overhead for $ N_x=201 $ would entail a 82418x82418 matrix, using 25 GB of memory. To store only the non-zero entries would cost around 1.8 MB)
	
	In practice to solve the system, we flatten each of the matrices corresponding to $ p_{i,j}^{n+1} $ and $ q_{i,j}^{n+1} $, and stack them on top of each other, to give the $ 2 (N_x+2)^2$ entry long vector needed. 
	
	Running the simulations with the Crank-Nicholson method unfortunately takes significantly longer time than using the forwards Euler method, since for large time steps Newtons method fails to converge for this problem (At least my implementation of it). There might be ways around this, for example using an adaptive time step size, where Newtons method is run with a large time step, and if it fails to converge, the time step is reduced until it converges. But that is outside my scope for this assignment.
	
	I do not know why the method fails for larger time steps (around 2 times those of the forward Euler method), as the Crank-Nicholson method is unconditionally stable.
	
	However, running the simulation until $ T=100 $ gives the results seen below in figures \ref{fig:CN1} and \ref{fig:CN2}:
	\begin{figure}[H]
		\centering
		\includegraphics[width=0.7\linewidth]{CN_K7.pdf}
		\includegraphics[width=0.7\linewidth]{CN_K8.pdf}
		\includegraphics[width=0.7\linewidth]{CN_K9.pdf}
		\caption{Crank-Nicholson integration for the Reaction Diffusion system. With parameters $ D_p=1, D_q = 8, C=4.5 $ and $ K \in \{7, 8, 9\} $}
		\label{fig:CN1}
	\end{figure}
	\begin{figure}[H]
		\centering
		\includegraphics[width=0.7\linewidth]{CN_K10.pdf}
		\includegraphics[width=0.7\linewidth]{CN_K11.pdf}
		\includegraphics[width=0.7\linewidth]{CN_K12.pdf}
		\caption{Crank-Nicholson integration for the Reaction Diffusion system. With parameters $ D_p=1, D_q = 8, C=4.5 $ and $ K \in \{10,11,12\} $}
		\label{fig:CN2}
	\end{figure}
\end{document}