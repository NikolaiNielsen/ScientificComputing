\documentclass[a4paper,10pt]{article}

\usepackage{Nikolai}
\usepackage[margin=0.75in]{geometry}

\title{Scientific Computing Project 1}
\author{Nikolai Plambech Nielsen\\LPK331}
\date{\today}


\begin{document}
	\maketitle
	\section{Questions for Week 1}
	\subsection*{A1}
	Solved in code. The first function of the file \texttt{functions.py}
	
	\subsection*{A2}
	For the three values of $ \omega \in \{1.300, 1.607, 3.000\} $, the condition number of the matrix $ \V{E}-\omega \V{S} $ under the max norm is tabulated in table \ref{tab:condition} (along with the results for the second part of the problem. See below)
	\begin{table}[H]
		\centering
		\begin{tabular}{c|c|c}
			$ \omega $ & $\text{cond}_{\infty}(\V{E}-\omega \V{S})$ & Valid significant digits\\
			\hline
			1.300 & 303.0742 & 5\\
			1.607 & 327825.2 & 2\\
			3.000 & 27.78252 & 6
		\end{tabular}
		\caption{Condition number for ($ \V{E}-\omega \V{S} $)}
		\label{tab:condition}
	\end{table}
	If everything except for the relative uncertainty in $ \V{z} $ is exact, then the relative uncertainty in the solution $ \V{x} $ is given by
	\begin{equation}\label{eq:cond}
		\Big|\Big|\frac{\Delta \V{x}}{\V{x}}\Big|\Big|_{\infty} = \text{cond}_{\infty} (\V{E}-\omega\V{S}) \Big|\Big|\frac{\Delta \V{z}}{\V{z}}\Big|\Big|_{\infty}
	\end{equation}
	which of course is the matrix equivalent to the scalar law. The difference here is that the matrix condition number has a lower bound of exactly 1 (then all column vectors are orthogonal to each other), whereas for a scalar we have a lower bound of 0.
	
	This also means that there is no way to ``conjure'' up more significant digits for a system of linear equations - we can only hope to keep the same amount of significant digits in the result as in the input.
	
	For this specific example we get the number of digits as the exponent on the r.h.s. of equation \ref{eq:cond}, which is essentially:
	\begin{equation}\label{eq:sigfig}
		\bigg\lfloor -\log_{10} \Big|\Big|\frac{\Delta \V{x}}{\V{x}}\Big|\Big|_{\infty}\bigg\rfloor
	\end{equation}
	The results are again tabulated in table \ref{tab:condition}.
	
	\subsection*{B1}
	Again, the number of valid significant digits is given by equation \ref{eq:sigfig}, but this time the relative error is
	\begin{equation}\label{key}
		\frac{||\Delta \V{x} ||_{\infty}}{||\hat{\V{x}} ||_{\infty}} \leq \text{cond}_{\infty} (\V{E} - \omega \V{S}) \frac{||\delta \omega \V{S}||_{\infty}}{||\V{E} - \omega \V{S}||_{\infty}}
	\end{equation}
	The results are tabulated below in \ref{tab:perturbed}:
	\begin{table}[H]
		\centering
		\begin{tabular}{c|c|c}
			$ \omega $ & relative error & valid significant digits \\
			\hline
			1.300 & 4.751$ \D 10^{-3} $ & 2\\
			1.607 & 5.090$ \D 10^0 $ & 0\\
			3.000 & 4.135 $ \D 10^{-4} $ & 3 
		\end{tabular}
		\caption{relative error in $ \V{x} $ with perturbed $ \omega $.}
		\label{tab:perturbed}
	\end{table}
	
	\subsection*{C}
	The functions are implemented (along with a wrapper function to actually solve the linear system) in \texttt{functions.py}. For the matrix system
	\begin{equation}\label{key}
		\begin{bmatrix}
		2 & 1 & 1 \\ 4 & 1 & 4 \\ -6 & -5 & 3
		\end{bmatrix} \begin{bmatrix}
		x_1 \\ x_2 \\ x_3
		\end{bmatrix} = \begin{bmatrix}
		4 \\ 11 \\ 4
		\end{bmatrix},
	\end{equation}
	the solution, found with my implemented function is
	\begin{equation}\label{key}
		\V{x} = \begin{bmatrix}
		-4 \\ 7 \\ 5
		\end{bmatrix},
	\end{equation}
	with a residual of
	\begin{equation}\label{key}
		||\V{x} - \V{x}_{\text{np}}||_2 = 5.704 \D 10^{-15},
	\end{equation}
	where $ \V{x}_{\text{np}} $ is the solution obtained with \texttt{np.linalg.solve}. As we see, the residual is on the order of 10$ \epsilon $.
	
	\subsection*{D1}
	The upper and lower bounds on $ \alpha(\omega\pm \delta \omega) $ for $ \omega \in \{1.300, 1.607, 3.000\} $ and $ \delta \omega = 0.5\D10^{-3} $ are tabulated in \ref{tab:bounds}
	\begin{table}[H]
		\centering
		\begin{tabular}{c|c|c|c|c}
			$ \omega $ & $ \alpha(\omega-\delta \omega) $ & $ \alpha(\omega) $ & $ \alpha(\omega+\delta \omega) $ & max($ ||\Delta\alpha||_{\infty}/||\alpha||_{\infty} $) \\
			\hline
			1.300 & -4.893 & -4.876 & -4.859 & 3.586$ \D10^{-3} $ \\
			1.607 & 151.096 & -434.961 & -91.094 & 1.347 $ \D10^0 $ \\
			3.000 & -2.905$ \D 10^{-1} $ & -2.904 $ \D 10^{-1} $ & -2.902 $ \D 10^{-1} $ & 4.006 $ \D 10^{-4} $
		\end{tabular}
		\caption{Upper and lower bounds on $ \alpha(\omega) $.}
		\label{tab:bounds}
	\end{table}
	
	\subsection*{D2}
	Both of the error bounds work since they are upper bounds, and all values fall within them. Though the bound from (b) is much closer to the calculated value.
	
	\subsection*{E1}
	\begin{figure}[H]
		\centering
		\includegraphics[width=0.5\linewidth]{polarizability.pdf}
		\caption{}
		\label{fig:polarizability}
	\end{figure}
	
	
	\subsection*{E2}
	Around the value $ \omega=1.60686978 $ the polarizability encounters a discontinuity (at least within the resolution of the system. This is also seen in the condition number
	\begin{equation}\label{key}
		\text{cond}_{\infty} (\V{E}-\omega \V{S}) = 1.734 \D10^{10}
	\end{equation}
	meaning the matrix approaches non-singularity at this value.
	
	\subsection*{F1}
	Implemented in code. To check that $ \V{Q}^T \V{Q} = \V{I} $ and $ \V{QR}=\V{A} $ I use \texttt{np.isclose(QR, A).all()} (and likewise for $ \V{Q}^T \V{Q}=\V{I} $). Both lines yield a true boolean.
	
	\subsection*{F2}
	Implemented in code. We get
	\begin{equation}\label{key}
		\tilde{\V{x}} = \begin{bmatrix}
		1236 \\ 1943 \\ 2416
		\end{bmatrix}
	\end{equation}
	
	\subsection*{G1}
	Since $ \alpha(\omega) $ has a singularity around $ \omega \approx 1.60686978 $, and the polynomial cannot represent singularities, I pick $ \omega_p = 1.6 $. This yields 143 data points to use in fitting.
	
	\subsection*{G2}
	For $ n=4 $ the parameters are
	\begin{table}[H]
		\centering
		\begin{tabular}{c|c|c|c|c}
		$ a_0 $ & $ a_1 $ & $ a_2 $ & $ a_3 $ & $ a_4 $ \\
		\hline
		-595.429 & 1010.84 & -633.747 & 171.610 & -16.6671
	\end{tabular}
	\end{table}
	
	\subsection*{G3}
	For $ n=6 $ the parameters are
	\begin{table}[H]
		\centering
		\begin{tabular}{c|c|c|c|c|c|c}
			$ a_0 $ & $ a_1 $ & $ a_2 $ & $ a_3 $ & $ a_4 $ & $ a_5 $ & $ a_6 $ \\
			\hline
			-2303.24 & 3642.50 & -522.166 & -2305.73 & 1841.10 & -558.004 & 61.698
		\end{tabular}
	\end{table}
	The relative absolute accuracies are plotted on the left in figure \ref{fig:error1}
	
	\subsection*{G4}
	The amount of correct digits varies with $ \omega $, and is calculated with equation \ref{eq:sigfig}. The result is plotted on the right in figure \eqref{fig:error1}
	\begin{figure}[H]
		\centering
		\includegraphics[width=0.7\linewidth]{g34.pdf}
		\caption{Left: Relative absolute error on the polynomial approximations of $ \alpha(\omega) $ in the range $ \omega \in [1.2, 1.6] $, with $ n\in \{4, 6\} $. Right: The number of correct significant digits in the approximations.}
		\label{fig:error1}
	\end{figure}
	Of course one cannot have a negative number of correct significant digits - the proper formula should really be the maximum of equation \eqref{eq:sigfig} and 0.
	
	
	\subsection*{H1}
	To get the approximating function on a linear form we multiply by the denominator and rearrange so that we have all unknowns on one side of the equations:
	\begin{equation}\label{key}
		Q(\omega) = \sum_{j=0}^{n} a_j \omega^j - Q(\omega) \sum_{j=1}^{n} b_j \omega^j
	\end{equation}
	Next we use $ Q \approx \alpha $ (by assumption) and use linear least squares on this model to get the best parameters. We then plug this into the original expression for $ Q $ to use the approximating function. For $ n=2 $ we get the parameters
	\begin{table}[H]
		\centering
		\begin{tabular}{c|c|c|c|c}
			$ a_0 $ & $ a_1 $ & $ a_2 $ & $ b_1 $ & $ b_2 $ \\
			\hline
			0.73245626 & -0.56503326 & 0.06227827 & -1.48274436 & 0.53544673
		\end{tabular}
	\end{table}
	The relative absolute error is plotted in blue on the left in figure \ref{fig:error2}, and the correct number of digits is plotted in blue on the right.
	
	\subsection*{H2}
	For $ n=4 $ we get
	\begin{table}[H]
		\centering
		\begin{tabular}{c|c|c|c|c|c|c|c|c}
			$ a_0 $ & $ a_1 $ & $ a_2 $ & $ a_3 $ & $ a_4 $ & $ b_1 $ & $ b_2 $ & $ b_3 $ & $ b_4 $ \\
			\hline
			1.714026 & -2.848737 & 1.0675 & 0.04153 & -0.00592 & -1.67121 & -0.17905 & 1.37720 & -0.53492
		\end{tabular}
	\end{table}
	The relative absolute error and number of correct digits is plotted in orange in figure \ref{fig:error2}.
	\begin{figure}[H]
		\centering
		\includegraphics[width=0.7\linewidth]{h.pdf}
		\caption{Left: Relative absolute error on the rational approximations of $ \alpha(\omega) $ in the range $ \omega \in [1.2, 4] $, with $ n\in \{2, 4\} $. Right: The number of correct significant digits in the approximations.}
		\label{fig:error2}
	\end{figure}
	
	
\end{document}