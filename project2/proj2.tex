\documentclass[a4paper,10pt]{article}

\usepackage{Nikolai}
\usepackage[margin=0.75in]{geometry}

\title{Scientific Computing Project 2}
\author{Nikolai Plambech Nielsen\\LPK331}
\date{\today}


\begin{document}
	\maketitle
	\section*{A}
	The code is seen in file \texttt{f2.py}. Using the function on $ \textbf{K} $ gives the Gershgorin disks seen in table \ref{tab:disks}
	\begin{table}[H]
		\centering
		\begin{tabular}{c|c}
			Center & Radius \\
			\hline
			22345.91813211761 & 90954.38918993676 \\
			20970.291359471106 & 111324.65280062004\\
			42687.082433334916 & 126764.0694086381\\
			26259.27613697939 & 94137.99241796896\\
			15959.10794172863 & 99958.28412923093\\
			53149.927128462994 & 171091.65853598804\\
			42173.70083943926 & 71865.46384544748\\
			31276.609541102425 & 124290.7338304566\\
			16126.984973178549 & 79552.99761063268\\
			39042.18505838955 & 129453.08357102858\\
			31503.391029932365 & 133585.4244613245\\
			18523.82363971299 & 82107.1199063304\\
			62872.789483587374 & 206937.40894591622\\
			19011.945981070516 & 76188.8354648638\\
			31881.247846857976 & 123877.63470683568
		\end{tabular}
	\caption{The Gershgorin disks for the matrix $ \textbf{K} $.}
	\label{tab:disks}
	\end{table}
	
	\section*{B}
	The convergence criterion chosen for the power iteration is
	\begin{equation}\label{key}
		\text{abs}\pp{\frac{\lambda_{n} - \lambda_{n-1}}{\lambda_{n-1}}} < \varepsilon
	\end{equation}
	where by default $ \varepsilon = 1\D 10^{-6} $, and the eigenvalue $ \lambda $ is calculated with the Rayleigh quotient. The eigenvalues, number of iterations before convergence and the Rayleigh residual of the example matrices and $ \V{K} $ are tabulated below in table \ref{tab:power_iter}
	\begin{table}[H]
		\centering
		\begin{tabular}{c|c|c|c}
			Matrix & Eigenvalue & Rayleigh Residual & Iterations \\
			\hline
			$ \textbf{A}_1 $ & 3.9999989184158142 & 0.0036011104009171287 & 10\\
			$ \textbf{A}_2 $ & 3.9999987743348386 & 0.0022124632616849476 & 7\\
			$ \textbf{A}_3 $ & 12.298959795018773 & 8.933595023604699e-06 & 14\\
			$ \textbf{A}_4 $ & 16.116844235105 & 1.139976430233264e-06 & 6\\
			$ \textbf{A}_5 $ & 68.64208108540393 & 1.2957189457185429e-06 & 6\\
			$ \textbf{A}_6 $ & 1.9999994028072163 & 0.001144211257731418 & 4\\
			$ \V{K} $ & 151362.6666519405 & 0.017048828321719074 & 30
		\end{tabular}
		\caption{Largest eigenvalue of the example matrices and $ \V{K} $ using power iteration, with Rayleigh residual and number of iterations until convergence shown.}
		\label{tab:power_iter}
	\end{table}
	For each of the tables a random starting vector is chosen with \texttt{np.random.uniform}
	
	\section*{C}
	The same convergence criterion is used for Rayleigh iteration as for power iteration. If we use the LU-solver for computing the solution, the method will not be robust for singular matrices. The shifts may make the matrix non-singular, allowing us to use the algorithm, but it probably should not be relied upon. As such we choose the QR-solver. The result are shown in table \ref{tab:rayleigh_iter}.
	
	\begin{table}[H]
		\centering
		\begin{tabular}{c|c|c|c}
			Matrix & Eigenvalue & Rayleigh Residual & Iterations \\
			\hline
			$ \textbf{A}_1 $ & 4.0 &1.4505701123219478e-09 & 3\\
			$ \textbf{A}_2 $ & 4.0 &2.8805418676599316e-09 & 2\\
			$ \textbf{A}_3 $ & 12.29895839097071 & 1.1234667099445444e-14 & 5\\
			$ \textbf{A}_4 $ & 16.116843969809032 & 3.156822789041486e-11 & 4\\
			$ \textbf{A}_5 $ & 68.64208073700239 & 1.7112135109858138e-13 & 10\\
			$ \textbf{A}_6 $ & 2.0000000000000004 & 5.103831345936687e-11 & 2
		\end{tabular}
		\caption{Largest eigenvalue of the example matrices using Rayleigh iteration, with Rayleigh residual and number of iterations until convergence also shown.}
		\label{tab:rayleigh_iter}
	\end{table}
	
	
	\section*{D}
	To calculate multiple eigenvalues of $ \V{K} $ we need to use 
	
	
	
	
	
\end{document}