\documentclass[a4paper,10pt]{article}

\usepackage{Nikolai}
\usepackage[margin=0.75in]{geometry}

\title{Scientific Computing Project 2}
\author{Nikolai Plambech Nielsen\\LPK331}
\date{\today}


\begin{document}
	\maketitle
	\section*{A}
	The code is seen in file \texttt{f2.py}. Using the function on $ \textbf{K} $ gives the Gershgorin disks seen in table \ref{tab:disks}
	\begin{table}[H]
		\centering
		\begin{tabular}{c|c}
			Center & Radius \\
			\hline
			
		\end{tabular}
	\caption{The Gershgorin disks for the matrix $ \textbf{K} $.}
	\label{tab:disks}
	\end{table}
	
	\section*{B}
	The convergence criterion chosen for the power iteration is
	\begin{equation}\label{key}
		\text{abs}\pp{\frac{\lambda_{n} - \lambda_{n-1}}{\lambda_{n-1}}} < \varepsilon
	\end{equation}
	where by default $ \varepsilon = 1\D 10^{-6} $. The eigenvalues, number of iterations before convergence and the Rayleigh residual of the example matrices and $ \V{K} $ are tabulated below in table \ref{tab:power_iter}
	\begin{table}[H]
		\centering
		\begin{tabular}{c|c|c|c}
			Matrix & Eigenvalue & Rayleigh Residual & Iterations \\
			\hline
			$ \textbf{A}_1 $ & & & \\
			$ \textbf{A}_2 $ & & & \\
			$ \textbf{A}_3 $ & & & \\
			$ \textbf{A}_4 $ & & & \\
			$ \textbf{A}_5 $ & & & \\
			$ \textbf{A}_6 $ & & & \\
			$ \V{K} $ & & &
		\end{tabular}
		\caption{Largest eigenvalue of the example matrices and $ \V{K} $, with Rayleigh residual and number of iterations until convergence shown.}
		\label{tab:power_iter}
	\end{table}
	
	\section*{C}
	
	
	
\end{document}