\documentclass[a4paper,10pt]{article}

\usepackage{Nikolai}
\usepackage[margin=0.75in]{geometry}

\title{Scientific Computing Project 3}
\author{Nikolai Plambech Nielsen\\LPK331}
\date{\today}


\begin{document}
	\maketitle
	
	\section*{Question 1}
	For the 12-6 Lennard-Jones Potential between two atoms, we can simply define our constants and do a straightforward calculation. In this case our problem is one-dimensional, and we just create a linearly spaced set of points between 1.65 and 10 using np.linspace, calculate the potential at each value in the set of points, and plot the result.
	
	For clarity I mask the result to have the points with V>0 appear blue and the points with V<0 appear orange. Further I plot a second set of axes with the potential between 3 and 5, which is the "interesting" area, with both a root and a minimum. This is seen in figure \ref{fig:q1}
	
	Lastly we use the timeit module to measure the time it takes to calculate the total potential for the Argon cloud. The average time it takes is $ 1.4702 \D 10^{-4} $ S.
	\begin{figure}[H]
		\centering
		\includegraphics[width=0.5\linewidth]{q1fig.pdf}
		\caption{12-6 Lennard Jones potential in two areas.}
		\label{fig:q1}
	\end{figure}


	\section*{Question 2}
	The analytical solution to this problem is given by
	\begin{equation}\label{key}
		r = \frac{A^{1/6}}{B^{1/6}} = \sigma
	\end{equation}
	For this question I have chosen to implement four different root-finding algorithms:
	\begin{itemize}
		\item Bisection
		\item Newton-Raphson (with a central difference approximation to the derivative)
		\item Secant
		\item Inverse quadratic
	\end{itemize}
	For each of these algorithms I measure the time it takes to find the root given some starting condition (based on 10000 trials):
	\begin{table}[H]
		\centering
		\begin{tabular}{c|c|c|c}
			Method & Initial conditions & Result & Average time \\
			\hline
			Bisection & $ a=2, b=4 $ & 3.40100 & $ 2.986\D10^{-5} $ \\
			Newton-Raphson & $ x_0 = 2 $ & 3.40100 & $ 3.241\D10^{-5} $ \\
			Secant & $ x_0 = 1.65, x_1 = 2 $ & 3.40100 & $ 2.110\D10^{-5} $ \\
			Inverse quadratic & $ a=1, b=2, c=3 $ & 3.40100 & $ 2.083\D10^{-5} $
		\end{tabular}
		\caption{Comparison of four 1D root finding methods.}
		\label{tab:roots}
	\end{table}
	For each of them we see that it indeed finds the root correctly, and all algorithms solve the problem in roughly the same time - with inverse quadratic method being just slightly faster than the others, but still on the same order of magnitude. 
	
	For this problem, of course, we are dealing with a simple root, which all of the algorithms are adept at handling. Had we encountered a multiple root, the bisection algorithm is thrown out the window (unless, of course, it is a saddle point), and one would probably have to use the inverse fractional method.
	
	\section*{Question 3}
	For my implementation I have chosen to use the Golden Section Search as the 1D minimizer. We are of course not guaranteed that the objective function is unimodal in the search direction - but it should converge on some minimum in the search interval.
	
	I calculate the gradient analytically, and for both this and the potential calculation I make use of the \texttt{pdist} and \texttt{cdist} functions from \texttt{scipy}, to calculate distances between atoms. 
	
	I normalize the gradient using the max-norm to avoid overflow errors in the line search, and adjust $ \beta $ accordingly by including a factor of the ratios of the max-norms between the two gradients.
	
	I have chosen to set the line search interval between 0 and 2. It does not make sense to have it be negative, since that would entail searching along the positive gradient. The other end of the interval is also chosen rather arbitrarily - I would rather have my algorithm take small steps (essentially limited to 2Å in each direction), than have the atoms shoot out to a distance they will not feasibly recover from.
	
	Further I reset the algorithm every $ n $ iterations, such that $ \beta_k = 0 $ for $ k \mod n  = 0  $, as suggested in the book. I have arbitrarily chosen $ n=10 $.
	
	With this I get the result shown in figure \ref{fig:q3fig}
	
	\begin{figure}[H]
		\centering
		\includegraphics[width=0.5\linewidth]{q3fig.pdf}
		\caption{Final position and graph of total potential as a function of iteration number}
		\label{fig:q3fig}
	\end{figure}

	The algorithm finds a potential of $ 2.933\D10^{5} $ in 14.95 seconds - not exactly impressive compared to Scipy which finds a potential of -177.7 in 14.65 seconds.
	
	A thing to note though, is that my solution retains the planar shape of the initial configurations - since all points in the initial configurations are coplanar, it makes sense that the configuration will keep this shape. This of course does not yield an optimal solution, which is evident from the solution given by Scipy.
	
	\section*{Question 4}
	For this question I implement the simulated annealing algorithm. The basis of it is that we take a random step in a random direction in phase space (denoted as a neighbour), compute its cost. If the cost of the neighbour is smaller than the cost of the original state we adopt this as the new state.
	
	If the cost is greater, then we let a roll of the dice decide. We calculate the acceptance probability as
	\begin{equation}\label{key}
		p = e^{(f(x_k) - f(x_{k+1}))/T}
	\end{equation}
	and if $ p>r $ where $ r $ is a random number between 0 and 1 (uniformly distributed) we still accept it as the new state. This allows the simulated annealing algorithm to ``climb hills'' and approach a global minimum.
	
	$ T $ in the equation is the ``temperature''. This is slowly decreased as the simulation runs, to make unfavourable jumps less desirable as the simulation progresses.
	
	For the simulation we use the following parameters: $ T_i = 1.0 $, $ T_f = 10^{-5} $, $\alpha = (T_f/T_i)^{1/N} $, $ N=10^5 $ and perform 100 iterations for each temperature. Further we let the neighbour step be between -0.1 and 0.1.
	
	Letting the simulation run (with the seed 42) yields a potential of $ 5.224 \D 10^{3} $ in 165.9 seconds. So a better result than the conjugate gradient, but at a significant time cost.
	
	The resulting position and costs can be seen in figure \ref{fig:q4fig}
	\begin{figure}[H]
		\centering
		\includegraphics[width=0.5\linewidth]{q4fig.pdf}
		\caption{Final position and graph of total potential as a function of iteration number}
		\label{fig:q4fig}
	\end{figure}
	The iteration number here does not mean the cost at \textit{every} iteration, but rather the cost at every point where we adopt a neighbour as the new state.
	
	This algorithm could maybe be coupled with the conjugate gradient method to introduce components to the gradient that are orthogonal to the plane of the initial state, yielding a more optimal solution. I unfortunately do not have time for this, but it would be interesting to investigate.
	
	
\end{document}